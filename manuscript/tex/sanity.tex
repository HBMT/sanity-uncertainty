\chapterNote{%
\shakerLightFont
{\fontsize{10}{17}\selectfont\itshape
Any speech which ignores uncertainty\\
is not the speech of a sage. Remember this.\\
Whatever we see or hear, be it pleasant or\\
sorrowful, just say, `This is not sure!'
\par}

\vspace*{4pt}

{\fontsize{8}{8}\selectfont AJAHN CHAH}%
\par
}

\chapter{Sanity in The Midst of Uncertainty}

\enlargethispage{\baselineskip}

Whilst visiting a local school some years ago, I asked a class of
teenage students if they could think of anything at all that was
genuinely unchanging, anything truly permanent. Encouraging them to
consider impermanence seemed like a suitable way to introduce them to
basic Buddhist teachings. I wasn't expecting to see any hands raised.
But one student did put his hand up and told me there was something
permanent, something truly unchanging, and that was the law of
impermanence itself. Whether that student was particularly discerning or
had been primed by his teacher in advance, I don't know, but it was good
to have the opportunity to speak in that environment about something as
important as the law of impermanence. The Buddha said that it is wisdom
that understands the law of impermanence, and it is this wisdom which
will protect us from undue suffering.

\enlargethispage{\baselineskip}

Change and the uncertain nature of things are nothing new. However, what
is new, thanks to technology, is the pace of change. And never before
have we had such ease of access to information about the changing nature
of things; on the microcosmic level, where we might study the dynamics
of nano particles; on the macrocosmic level, where we can learn about
expanding and contracting universes; and so much more in between.

\clearpage

Naturally we feel grateful for the many advantages technology has
provided, such as possibilities for better health care, improved
education and a safer society. However, this rapid rate of change is
undeniably contributing to a significant increase in the level of
collective anxiety. Many of the structures in society which previously
provided a relative sense of security now seem less reliable. To name
just a few examples: some multinational corporations are even more
powerful than elected governments; mass migration of refugees has led to
complex difficulties never seen before; block-chain technology is
undermining traditional ways of doing business; and the indiscriminate
use of antibiotics has created superbugs.

But as we enter this contemplation on uncertainty, let's be careful that
we don't assume change itself is necessarily a problem. People living
under totalitarian dictatorships live in hope that someday soon things
will change. If we are suffering from an illness, we hope that the
symptoms will soon change and we will recover. In Japanese culture the
perception of impermanence is central to several art forms: for
centuries exquisite poetry has been written about the falling of cherry
blossoms; the philosophy of \emph{Wabi-sabi} celebrates `the fortunate
accident'; \emph{Kintsukuroi} highlights cracks in a broken pot which
have been mended using gold.

Sadly though, it is also true that technologically advanced Japan has
one of the highest rates of suicide in the world. According to WHO
statistics,\cite{who-suicide} over the last 45 years there has been a 60\%
worldwide increase in suicide. This rate is due to double again by
2020. It is clear that no amount of technological advancement, ready
access to information, or sublime forms of art is enough to save us from
the suffering that arises from fear of uncertainty. While technology
itself is neutral, it does have the effect of amplifying whatever it
touches. This in turn leads to intensification, and regarding the rate
of change, is contributing to people feeling pushed beyond what they can
tolerate.

\section{Strategic Optimism}

From all the indicators, it would appear that this accelerating rate of
change is not going to cease any time soon. For many, unfortunately,
this uncertainty leads to becoming habitually pessimistic. However, from
the perspective of a commitment to serving reality, our reactions to
change and uncertainty need not intimidate us; instead they need to be
studied and understood. Allowing ourselves to be defined by feelings of
pessimism is not an obligation; it is a choice, even when it doesn't
feel that way.

\clearpage

It seems to me that the approach most conducive to progress on the path,
and the most skilful way of dealing with feelings of fear is that of
strategic optimism. When people ask me how I personally deal with
challenging dilemmas, I often tell them that I am a strategic optimist.
Of course, naive optimism is very dangerous, as is habitual pessimism.
Both these perspectives blind us to a great many possibilities. But when
our decision to intentionally develop an optimistic attitude is informed
by mindfulness, sense restraint and skilful reflection, it is neither
naive nor dangerous. We can chose to adopt such an approach out of a
desire to take full responsibility for our actions of body, speech and
mind, so as to do more than merely react out of conditioned preference.
If we are honest with ourselves, we have to admit that we don't
\emph{know} that everything is getting worse, any more than we
\emph{know} that everything is getting better. But what we can observe
is how being caught in negative mind-states affects the way things
unfold and, conversely, how cultivating wholesome mind-states can have a
positive influence. It doesn't take a lot of study to see that being
positively disposed towards the results of our efforts brings beneficial
results.

I realize that, to some, speaking this way will sound idealistic, but I
am not talking about how things \emph{should} be, but about how things
are. How are we relating to reality moment by moment? Are we interested
in what works, what helps, or are we merely caught in negative thinking
and worry about what could happen in the future?

Thinking is something most of us are quite good at. We tend to do a lot
of it. Not-thinking, on the other hand, is probably something we are not
so good at and which would benefit from more attention. If talking about
paying attention to not-thinking sounds contradictory, that is because
so much of the time our attention is tethered to the activity of
thinking. But it doesn't have to be that way. Using the various
meditation techniques, we can discipline attention to untether attention
from thinking. Then we are able to pay attention in a more feeling way
-- a feeling investigation. Investigating using concepts is powerful,
but, as with technology, it has its limitations. When we develop the
ability to investigate without the persistent interruption of mental
verbiage, we will have access to a different quality of discernment,
where discriminative intelligence and intuition can work creatively
together as partners, untangling our confusion.

\clearpage\mbox{}\thispagestyle{empty}
\photoFullBleed{02-two-swans.jpg}
\clearpage

\section{A Trusting Approach}

This type of investigation does require a willingness to trust in our
intention, in our sincere interest to discover truth. We have to get
used to letting go of craving for certainty, and come to terms with what
uncertainty actually feels like in the whole body-mind. The temptation
to turn away from such unpleasant feelings can be strong. But if we
inhibit our reactions, again using mindfulness, sense restraint and
skilful reflection, it is possible that we won't have to turn away, but
instead enter into a realistic relationship with what we feel is
challenging us. This approach applies to everything we encounter on this
spiritual journey but here we are specifically contemplating our
reactions when faced with the unknown.

If we could recall how as a child we learnt to walk, we'd probably
realize we didn't think so much about it, we just did it. Later on, when
we were learning to ride a bicycle, it wasn't thinking that taught us.
Learning life's lessons requires a desire to succeed at them, and trust
that success can be achieved. It also takes quite a bit of falling over
and getting up again. But many of these tasks don't necessarily depend
on thinking. Admittedly, with learning to walk and ride a bike we had
the encouraging example of others who were already ably doing these
things. In the case of learning to trust that we can accommodate intense
fear of uncertainty, there are maybe not so many examples of mature
competence around, but that doesn't mean we must assume it can't be
done. In the material world there are many inspiring examples of those
who have dared to go into the unknown, motivated by trust in their own
convictions.

\section{Unrealistic Expectations}

Nobody likes feeling insecure; all beings want to feel safe. Where,
then, should we be looking to find real security? How are we to live
with an awareness of the changing nature of conditioned existence? The
Buddha advises us that regardless of how daunting fear and uncertainty
might appear, we should train ourselves to trust in wisdom. Wisdom is
what sees beyond the way things merely appear to be and discerns
actuality. If we learn to trust that there is real wisdom, and in its
transformative power, that trust can help protect us from falling into
despair.

And this wisdom is more than a conceptual level of understanding. The
Buddha was talking about a quality of insight that transforms the
gnawing feelings of fear of insecurity into clarity and calm. With that
clarity comes the ability to accord with any amount of uncertainty. The
perception of uncertainty does not have to turn into feelings of
insecurity. In other words, the reliable sense of security that we seek
is to be found in unobstructed awareness. So long as we don't have such
a quality of awareness, such a level of understanding, we will suffer
from expecting life to be something which it is not and never can be.

In the early 1970's the wife of the former US ambassador to Thailand was
a frequent visitor to a temple in Bangkok where His Majesty the King of
Thailand had spent time as a monk. She was a dedicated disciple of the
abbot there, Venerable Somdet Nyanasamvara, and regularly came to listen
to his Dhamma teachings. One day she spoke with the abbot about the
uninspiring state of so many of the monasteries which she had seen. How
could it be, she wondered, that when the Dhamma is so beautiful and so
precious, there are so many unkempt monasteries, and, for that matter so
many unruly monks? Many of the monasteries had crumbling buildings and
were noisy and virtually overrun with dogs suffering from mange. The
venerable abbot listened attentively to what she had to say and then
quietly replied,

\begin{quote}
\itshape
`While it is true the Dhamma is timeless and precious, the institutions
and structures of Buddhism are subject to the law of impermanence, as are
all compounded things.'
\end{quote}

That was wisdom speaking. The abbot was not saying we shouldn't try to do
something about the dysfunctional state of institutional Buddhism, but was
placing emphasis on the fundamental right view upon which all other
considerations need to be based.

\clearpage\mbox{}\thispagestyle{empty}
\photoFullBleed{03-hole-in-the-wall.jpg}
\clearpage

When we don't have enough clarity and calm, we easily fall prey to
misperceiving that which is in front of us. We attempt to find security
within that which is insecure. We try to find stability within that
which is unstable. On one occasion early in my monastic training, when I
was caught up even more than usual in doubt and despair, I went to see
Ajahn Chah, hoping he might help relieve me of my misery. He kindly
listened to my worries and then commented:

\begin{quote}
\itshape
All these worries and doubts that you have are about things that are
uncertain. What do you think happens when you try to make something that is
inherently uncertain certain? You create suffering.
\end{quote}

Again, that was wisdom speaking. And it definitely made a difference. When we
don't have enough wisdom ourselves, we need to borrow some wisdom from others.
Not that all my suffering suddenly disappeared, but it did become more workable.
It also helped enormously when Ajahn Chah spoke of some of the ordeals he had
endured as he struggled to come to terms with doubt and worry. We benefit from
knowing we are not alone in our struggles.

To train our whole body and mind in awareness of the experience of
uncertainty is to cultivate wisdom and take responsibility for our
reactions to life. If we insist on maintaining our habits of resistance
and avoiding how we feel in the face of uncertainty, we perpetuate
suffering. Wisdom recognizes the many tricks we get up to and the
stories we tell ourselves. And it is in such recognition that letting go
happens.

\clearpage

Dhammapada verse 277 says:

\begin{quote}
{\itshape `All conditioned things are impermanent;\\
when we see this with wisdom\\
we will tire of this life of suffering.\\
This is the Way to purification.'}\thinspace \cite{dhammapada}
\end{quote}

First, the Buddha is stating the truth of our situation as he has
realized it: everything is in a state of flux. Next he tells us what we
need to do about it: to train our faculties until we see this truth
clearly for ourselves. Then he points to the result: that we will tire
of this life of suffering which we are creating for ourselves. This last
point introduces a particularly important aspect of the spiritual
journey. There is a positive emotion sometimes overlooked by
Westerners, called \emph{nibbidā} in Pali, which we could translate as
disenchantment. Our fondness for excitement can result in our failing to
appreciate how agreeable it feels to not be excited. We are so used
to being intoxicated with agitated feelings of excitement. Our failure
is reinforced by all those around us who are similarly committed to
being distracted by excitement. If wisdom, not the pursuit of happiness,
is our priority, it helps to become familiar with this quieter, cooler
mood. It can be likened to having eaten a meal until we felt full and
then having more food placed in front of us. Or perhaps a cool breeze in
the evening at the end of a long hot day. This coolness is something
like what the Buddha says happens when we investigate impermanence and
begin to see the world with wisdom. It is similar to boredom, but
without the negativity.

Some might ask why we pay so much attention to these ancient teachings
given 2600 years ago, when what we need to be dealing with are the
challenges that we face here and now. The fact that the
phases of the moon were deciphered long ago does not make them any less
true today. The Buddha's insight into the truth of the impermanence of
all conditioned things is as relevant now as it was when he realized it.
He didn't invent the law of impermanence; he identified its importance
and taught how not knowing this truth is one of the main reasons for our
struggle to cope with uncertainty. It is wisdom which recognizes the law
of cause and effect; wisdom sees the causes of suffering and the
beneficial effects of letting go of those causes.

\section{Wisdom Culture}

On this journey of awakening we can expect to encounter many struggles.
When we commit ourselves to serving reality and no longer serving the
world of deluded personality, we are guaranteeing an ordeal for
ourselves. The views with which most of us were conditioned in our early
life do not accord with actuality, and we have to work hard to be freed
from those views. For example, we were taught at least implicitly if not
explicitly, that we own our bodies, when in truth they belong to nature.
If we really owned our bodies, we would not have them become old, sick
and ugly. Likewise, we were taught that the pursuit of happiness is a
genuinely worthwhile endeavour. In truth, unless we have wisdom, when we
do experience happiness we cling to it and sow the seeds for further
unhappiness. Very few of us were taught that what is truly worthwhile is
the pursuit of wisdom; that which sees clearly the relativity of
happiness and unhappiness, and knows how to accord with the changing
nature of all things.

\clearpage

When there is wisdom, there is flexibility. In any given situation there
is the ability to view what is gained and what is lost; not just one
perspective. If there is wisdom, there is the ability to adjust
according to what is needed, rather than simply clinging to a fixed
position because it suits our preferences; wisdom knows that all
preferences are relative. Wisdom produces the kind of flexibility that
conduces to well-being for oneself and for others, not the kind of
flexibility which means putting our personal preferences ahead of
everything else.

The Chinese meditation master Venerable Hsuan Hua had a helpful way of
summarizing spiritual practice:

\begin{quote}
 \itshape
 `We need to be able to accord with conditions without compromising principles.'
\end{quote}

Maybe we know people who
insist on holding fast to their principles, but have difficulty adapting
when flexibility is called for; often it is not very comfortable to be
around such people. Maybe we also know those who are quick to `accord
with conditions' but perhaps not very dependable when it comes to
honouring true principles. It takes a maturity of embodied wisdom to
meet all the experiences which life gives us, without having our
conditioned preferences dictate how we respond.

So the solution to the predicament in which we find ourselves, of having
to cope with an ever accelerating rate of change, is not to be found in
judging our struggles or despairing over them, but in becoming
interested in the actual causes of our struggles. And this means having
the awareness to see just where, when and how we resist reality.

\section{What If We Still Can't Cope?}

One of the advantages of living in this age of advanced technology is
the access we have to teachings from the various spiritual traditions.
It is not just information about the dynamics of nano particles or
expanding and contracting universes that we can access; never before
have has it been so easy to obtain Dhamma books and listen to Dhamma
talks.

However, what if we find that all these wise words are not enough? What
do we do if after listening to many hours of talks, and sitting for many
hours, weeks or months in meditation, we still find we are struggling to
cope with feelings of anxiety and fear? Is it truly the case that this
set of spiritual exercises we have inherited is enough to free us from
suffering?

\enlargethispage{\baselineskip}

In an incident recorded in the traditional Buddhist scriptures,\cite{hunger}
the Buddha saw that the mind of a certain villager was ripe for
understanding Dhamma, so he travelled to where the villager was living
to teach him. When he arrived at the village, he saw that this young man
was tired and hungry from the hard physical work he had been doing.
Before offering him teachings, the Buddha had the local elders make sure
the man was properly fed. After having eaten and then listened to the
Buddha's teachings, this man awakened to Dhamma. Later on, when a group
of monks were discussing what had just happened, the Buddha explained to
them that feelings of hunger can hinder somebody's potential for
awakening. Most of us don't suffer the pain of physical hunger, but many
of us do suffer from a sort of mental hunger. If our fundamental
psychological needs have not been adequately met, we can suffer a
terrible inner sense of lack which can similarly be an obstacle on the
path.

\clearpage

\enlargethispage{\baselineskip}

Recently a young monk from Thailand stayed with us for a few days. His
grasp of the English language was just about good enough to engage in
conversation. One day, with a somewhat perplexed expression on his face,
he told us that he had been speaking with a female guest about what led
her to visit the monastery. She had told him that she was looking for
happiness. The monk seemed genuinely puzzled as to why she felt she had
to go somewhere and do something special to find happiness. From what he
told us about his personal experience, it seemed that whenever he had
difficulties, he would just sit down and stop doing whatever he had been
doing, until he had comfortably reconnected with an inner stillness
which was consistently associated with feeling happy. Presumably he
thought everybody could do that; he said that he had never had to go
anywhere or do anything special to find happiness. Happiness had always
been there whenever he stopped being busy doing things.

I wonder how many of us find this to be the case. Do we live our lives
confidently, knowing that however chaotic and uncertain things might
appear, we can always just sit down and wait for a few minutes until
self-existent happiness re-emerges, and then be bathed in well-being? It
is more likely that we have spent years struggling with self-criticism
over our hyper-active minds which refuse to settle; over feelings of low
self-esteem; over compulsive worry and doubt. What must it be like to
have a nervous system equipped with a sense that the possibility of
refreshing and renewing is always available? Compare that with a nervous
system which rarely refreshes, but instead stores up stress and an
ever-increasing sense of pending doom.

\enlargethispage{\baselineskip}

On the outside we humans are all more or less the same; we are born,
grown old, fall sick and die. But how we view life and how we process
experience are not the same.

A fifteen-year-old computer operating a
dial-up modem does not have the same processing power as a brand new
computer connected to fibre-optic broadband. They might look somewhat
similar, but their inner functioning is very different. The process of
conditioning that many of us have gone through in our culture has led to
ego structures shaped very differently from that of the young Thai monk.
When we stop being busy with what we have been doing outwardly, we are
very likely to encounter inner currents of busy-ness. We should
therefore expect that as we learn to navigate the path of freedom from
suffering, there will be times when we need to adapt some of the
practices we have inherited.

\section{Handling Old Pain}

The mental pain which some people have to endure can be even worse than
physical torment. We should consider carefully whether the spiritual
techniques that we pick up are in fact designed to address disruptive
mental turmoil. We wouldn't, for instance, encourage someone to go and
see a dietician if we knew that they were recovering from a broken leg
and what they needed was physiotherapy. When the Buddha taught about
overcoming the Five Hindrances,\cite{hindrances} I don't think he was referring
to dealing with an intensely painful memory of abuse suffered as a
child; I suspect he was alluding to a rather more refined level of
enquiry. So what do we do if we are overwhelmed by old pain that we
unearth in our practice?

In many meditation centres there is a culture which encourages not
needing anybody or anything other than a passionate commitment to the
meditation technique. I~remember a notice nailed to a tree at Ajahn
Chah's monastery that stated: Eat little, Sleep little, Speak little.
However, I know Ajahn Chah also told overly idealistic Westerners that
they should eat more. And there can definitely be times in practice when
we should sleep more. And sometimes what is needed is to speak more.
Desperately clinging to principles and not being able to `accord with
conditions' is not the way. The way is what really works. If what is
needed is to speak with someone with the skills to help us make sense of
our confusion, then what we should do is speak.

What I am referring to here is meditators using psychotherapy. Not so
many years ago, the mention of the word 'psychotherapy' in the context
of a Buddhist meditation centre or monastery was almost heretical. I
have heard the opposite was also true: mention within some
psychotherapeutic circles of Buddhist teachings on selflessness
(\emph{anatta}) was completely taboo. These days it seems that both
parties are a bit better informed of how different skills and practices
are designed to serve different purposes. A good enough sense of
self-confidence is necessary to be able to find our way around in this
world, and psychotherapy can be helpful in establishing that good enough
level of confidence. But a conventional sense of confidence and
happiness does not mean we will have calm and equanimity when it comes
to handling strong feelings of insecurity, or, for that matter, the
inevitability of our own death. That takes wisdom, or a transcendent
level of understanding. This is where the tools and techniques preserved
within the wisdom traditions are most helpful.

Returning to our question about what to do if we come across old pain
that is so intense that we find ourselves really struggling to cope: the
first thing to check is whether our commitment to observing moral
precepts is in order. Are we living in ways that accord with integrity
and will give rise to self-respect? If our conduct of body and speech is
appropriate, we then need to check whether we are getting enough
physical exercise. Sometimes vigorous physical activity can help us deal
with strong emotions. Fear and anger in particular can send hormones
racing around our bodies, and if we are sitting all day, these chemicals
can turn toxic. Exercising until we feel tired can be very grounding.

Then there is the matter of what we eat and drink. Being too idealistic
and not getting enough of the right kind of nourishment can exhaust our
nervous system to the point where we won't have the stamina to deal with
the onslaught of strong emotions. Also, eating too much food, especially
sugar, can lead to imbalances that mean we can't accurately read where
our energy is coming from: is it authentic or synthetic? Are our eating
habits skilfully considered or do we use food as a distraction?

If after checking that we are doing what we can on the physical level,
we are still struggling to come to terms with inner chaos, the next
thing to do is ask for help. The ability to ask for help at the time it
is needed is tremendously important. There is a sizeable body of work
comparing male and female suicide rates. Surely it is not insignificant
that the rate of male suicide is so much higher than that of females.\cite{suicide-rates}
And surely the fact that many men seem to find it difficult to
ask for help must be a factor. How much of that tendency in men is
nature and how much is nurture is an ongoing debate, but the fact
remains that the inability to ask for help when it is needed is
definitely a disadvantage.

\clearpage

Not everyone knows the feeling of needing, or even wanting, help. There
are some who experience memories of intense pain, but find they can
resolve them without the specific support of others. But there are
others who may never reach resolution unless they have help. What
matters is not allowing fixed views about whether we should or shouldn't
need help to get in the way.

Sometimes I am asked by meditators how to approach a therapist. My
recommendation is first to find out if the therapist's life has a
spiritual foundation. It need not necessarily be the same tradition as
the meditator's. What matters is that the therapist knows deeply that
they are accountable to a higher authority. Therapists who don't have
confidence in a reality beyond their own personality are dangerously
vulnerable to ego inflation. Most of the world's major religions offer
their followers reminders that their egos are not the centre of
existence. In so doing they provide them with a degree of protection
against becoming completely identified as the ego. This aspect of the
relationship with a therapist also applies to that stage in the therapy
when the client reaches a new quality of contentment and ease. Welcome
as this new-found happiness may well be, it is important to remember
this is still at the level of personality and not to become intoxicated
by the sense of relief, forgetting the commitment to the spiritual
journey. This can happen, but the risk is better managed when the
therapist and client share an appreciation for the spiritual dimension
of life.

Secondly, it is essential for the client and the therapist to respect
each other, and that this respect generates an atmosphere of trust.
Possibly the quality of the relationship is even more important than
whether the therapist comes from a behavioural, cognitive, humanistic or
any other psychotherapeutic tradition. From what I have observed, it is
the relationship that precipitates the healing. Of course, I am not
suggesting that all schools of psychotherapy are the same; they are not.
Some will specialize in dealing with difficulties arising from trauma
suffered at an early stage of development, while others are better
equipped to address issues that arise from trauma that occurred at a
later stage.

\enlargethispage{\baselineskip}

Then there is the question of whether to use talking therapy or touch
therapy. By touch therapy I am referring to such disciplines as
craniosacral therapy, the Alexander Technique or Shiatsu. This might be
a case of trial and error. We discussed earlier in this contemplation
how to bring discriminative intelligence and intuition together to
untangle our confusion, and this could be an opportunity to experiment.
Just as a cook knows if the food has the right amount of salt by tasting
it, a meditator determines whether the work with the therapist is
beneficial or not by feeling the consequence in the whole body-mind.

Meditators who decide to engage a therapist to support them on their
journey do well to remind themselves regularly that they are allowed to
be asking for this support. They are not letting down the team.
Unfortunately, shaming still occurs in some communities. Some people are
simply not adequately informed about the various skilful means needed to
deal with obstructions on the path. Hence it is wise to choose carefully
those with whom we might discuss these matters.

And it is always wise to remember that it is OK not to know. If we knew
how to stop suffering, we would stop it immediately. Our practice is
founded on faith in true principles and the skilful effort to accord
with the changing conditions in which we live.

When Ajahn Chah died in 1992, we set up a special shrine here in the Dhamma Hall
to mark his passing and to honour his life. At the centre of that
shrine was a portrait of our teacher, lit by a standard lamp which was
left turned on all day and all night. After the traditional seven-day
period of remembrance, a large gathering met in the hall to reflect on
his life and express gratitude for the many gifts he had given us.
During that service I read a translation of one of his talks called
\emph{Not Sure}.\thinspace \cite{not-sure} The quote at the beginning of this chapter is
an extract from that teaching. I came to the words,

\begin{quote}
\itshape
Any speech which ignores uncertainty is not the speech of a sage.
\end{quote}

Then the light bulb in the standard lamp blew out. I~am not suggesting that we
should read too much into that occurrence, but we would be wise to note how
uncertainty and impermanence are constantly being revealed to us.

Earlier I referred to approaching life's challenges from the disposition
of a strategic optimist. Another way of talking about strategic optimism
is being hopeful. Hopefulness is a form of creative vigilance. When hope
is absent we are prone to feeling hopeless, which in turn can lead to
depression and despair. If we are hopeful without being mindful, we are
easily tricked into having unrealistic expectations of life. But if we
can maintain a positive attitude and at the same time embrace
uncertainty, we will be protected from collapsing into despair. Hope,
mindfulness and an interest in what is real, support the cultivation of
the wisdom which sees the way through confusion.

I hope this contemplation has been helpful.

Thank you very much for your attention.

